\documentclass[12pt, a4paper]{article}
\usepackage[verbose,left=25mm,right=25mm,top=35mm,bottom=30mm]{geometry}
\usepackage{algorithm}
\usepackage[noend]{algpseudocode}
\usepackage{hyperref}
\usepackage{booktabs}
\usepackage{amsmath, amssymb, mathtools}
\usepackage{subcaption}
\usepackage[pdftex]{color,graphicx}
\renewcommand{\algorithmicrequire}{\textbf{Input:}}
\renewcommand{\algorithmicensure}{\textbf{Output:}}

\usepackage{multirow}
%\geometry{right=2cm,left=2cm,bottom=2.5cm,top=2.5cm}

\input{configs.tex}

\newtheorem{prop}{Proposição}[section]
\newtheorem{teo}{Teorema}[section]
\newtheorem{defi}{Definição}[section]
\newtheorem{lema}{Lema}[section]

\begin{document}

\begin{titlepage}
	\begin{center}
    	\large{Universidade Federal de São Carlos}\\
    	\vspace{65pt}
    	\vspace{35pt}
    	\large{Processo FAPESP \#2020/06103-0}\\ 
        \vspace{95pt}
        \textbf{\LARGE{Heurísticas e Meta-Heurísticas para Problemas de Roteamento de Veículos com Drones}}\\
        \vspace{0,5cm}
        \textbf{Relatório Parcial}\\
        \vspace{2,5cm}
        Orientador: Prof. Dr. Mário César San Felice
            
        Bolsista: Matheus Teixeira Mattioli
    	\vspace{2,5cm}
	\end{center}
	
	\begin{flushleft}
		\begin{tabbing}
			
	\end{tabbing}
 \end{flushleft}
	\vspace{1cm}
	
	\begin{center}
		\vspace{\fill}
		 Agosto de 2021 a Janeiro de 2022\\
		 São Carlos - SP, Brasil
			\end{center}
\end{titlepage}

\newpage
\tableofcontents
\thispagestyle{empty}
\newpage



\section{Introdução} \label{sec:introduction}
% Falar sobre a área de especialização da ic e dar como exemplo o problema estudado
% Trazer o interesse comercial
Neste relatório parcial vamos tratar de problemas de otimização combinatória, área da computação responsável por estudar problemas em que é necessário tomar decisões que maximizem ou minimizem uma função objetivo. Em particular, escolhemos problemas de roteamento de veículos com drones, nos quais devemos traçar rotas conectando pontos de interesse por meio de veículos tripulados (caminhões) e veículos não tripulados (drones). Apesar dessa definição simples, na prática este problema pode ser aplicado para uma gama muito grande de serviços, por exemplo, podemos realizar monitoramento de fronteiras ou áreas florestais, realizar serviços urbanos, como leitura de consumo de água e energia, ou entrega de produtos à clientes. Este último, em particular interessa a grandes empresas como a Amazon, segundo Shiva et al.~\cite{singireddy2018technology}.\par
% Interesse academico e dificuldade, tentar justificar o uso de heurísticas
Este problema também desperta interesses acadêmicos, pois além  das dificuldades inerentes ao VRP e TSP, existe a complexidade de se trabalhar com drones, conforme apontam Sugihara e Gupta~\cite{sugihara2011path}. Algumas dessas dificuldades são o tempo de duração da bateria dos drones e a carga que o drone aguenta transportar. Assim, ao combinar os dois tipos de transporte, temos um problema bastante desafiador. Para resolver essa situação, muitos pesquisadores apostaram no uso de heurísticas e meta-heurísticas, obtivendo resultados interessantes, conforme vemos nos artigos de Quang Minh et al.~\cite{ha2018min}, Murray~\cite{murray2015flying}, Sundar~\cite{sundar2013algorithms} e Freitas et al.~\cite{de2020variable}. Por conta disso, resolvemos utilizar heurísticas e metaheurísticas para atacar problemas de roteamento de drones nesta Iniciação Científica.\par
% Falar sobre as variantes de vrpd e sobre o levantamento bibliográfico que fizemos
Problemas de roteamento com drones podem ser encontrados com muitos nomes distintos na literatura e possuem diversas variantes. Por conta disso, resolvemos realizar um levantamento bibliográfico. O survey de Khoufi et al.~\cite{khoufi2019survey} nos conta sobre problemas de \textit{Path Planning} envolvendo drones, tais como monitoramento de grandes áreas por drones utilizando bases de recarregamento de bateria. Já o survey de Mozaffari et al.~\cite{mozaffari2019tutorial} nos conta, principalmente, sobre os benefícios do uso de drones para comunicação wireless, visando manutenção de redes ou coleta de informações. Neste contexto o drone é comumente chamado de \textit{data mule}. Nas próximas seções descreveremos com mais detalhes a classificação encontrada para estes problemas de roteamento envolvendo drones. \par
% Divisão do artigo
Nesse relatório serão apresentadas as atividades realizadas na primeira metade do projeto. Na Seção~\ref{sec:hier} será descrito a organização de problemas com drones encontrada. A Seção~\ref{sec:problem} descreve o problema que pretendemos atacar. A Seção~\ref{sec:algorithms} descreve heurísticas e metaheurísticas que estudamos e pretendemos utilizar. A Seção~\ref{sec:crono} apresenta o cronograma da iniciação científica. Por fim, a Seção~\ref{sec:concl} conclui o relatório técnico, explicitando os principais aprendizados deste primeiro semestre de iniciação científica.\par



\section{Hierarquia de Problemas com Drones}\label{sec:hier}

Nesta seção vamos classificar os problemas de roteamento envolvendo drones conforme estudados nos surveys citados anteriormente. Dessa forma, temos a seguinte hierarquia de problemas com drones:


% Quando retirar a cor lembrar de colocar ``~" antes da citaçao.
\begin{description}
	\item [Otimização de trajetória (\textit{Trajectory Optimization}).] Consiste em identificar uma trajetória que minimize alguma medida de desempenho, como tempo de voo ou consumo de combustível, enquanto satisfaz um conjunto de restrições cinemáticas, por exemplo, posição, velocidade e aceleração, e restrições dinâmicas, como forças e momento do veículo~\cite{coutinho2018unmanned}.
	
	\item [Planejamento de rota (\textit{Path Planning}).] Reduz o nível de detalhes físicos, por exemplo, ignora a dinâmica do veículo, mas ainda leva em consideração a posição e geometria dos drones e objetos do ambiente\cite{radmanesh2018overview, sarath2018prototype}. Ocasionalmente, variantes do \textit{Path Planning} podem conservar alguma dinâmica do veículo, por exemplo, restrições de movimento ou vento~\cite{macharet2018survey, luo2018integrated}.
	
	\item [Problemas de Roteamento de Drones (\textit{UAV\footnotemark Routing Problems}).]\footnotetext{Unmanned Aerial Vehicle.} Caixeiro viajante (TSP), Roteamento de veículos (VRP) e variantes envolvendo drones.
	\begin{description}
		\item [Atribuição de tarefas à drones (\textit{UAV Task Assignment}).] É um subproblema de Problemas de Roteamento, em que se deseja encontrar a atribuição ideal de tarefas em vez de atribuir mercadorias~\cite{jiang2017method}. Permite múltiplas visitas ou subtours.
		
		\item [TSP com drones (TSP-D).] Categorias de TSP-D são descritas a seguir:
		\begin{description}
			\item [Entrega-Transporte (\textit{Delivery-Transportation}).] Realizamos o roteamento de veículos e drones. Mais especificamente, roteamos o veículo tripulado até certo ponto e, nesse ponto, lançamos o drone para realizar tarefas em outros nós que o veículo não visita. Também conhecido como \textit{last-mile delivery}. Pode ser Delivery 1-Truck 1-Drone~\cite{agatz2018optimization}, 1-Truck m-Drone~\cite{tu2018traveling}, n-Truck m-Drone~\cite{kitjacharoenchai2019multiple}, de acordo com o número de caminhões e drones envolvidos.
			
			\item [\textit{Delay Tolerant Networks} (DTN).] As DTN são redes esparsas onde a comunicação direta de ponta a ponta completa entre origem e destino raramente pode ser estabelecida~\cite{barroca2018improving}. Os mecanismos de roteamento em DTNs contam com a mobilidade dos nós para conectar nós desconectados, transportando mensagens pela rede para superar a desconexão do caminho. A abordagem DTN consiste em introduzir nós dedicados em estabelecer comunicação entre nós comuns e aliviá-los de trabalhos que consomem energia, como roteamento e encaminhamento de mensagens. Os Veículos Aéreos Não Tripulados (UAVs) têm muitas aplicações potenciais em sistemas de comunicação sem fio, como fornecer conectividade sem fio econômica para dispositivos sem cobertura de infraestrutura devido, por exemplo, ao sombreamento intenso por terreno urbano ou montanhoso, ou danos à infraestrutura de comunicação devido a desastres naturais. Neste contexto, os UAVs podem ser nomeados de Data Mule.
			
			\item [Inteligência, Vigilância e Reconhecimento.] Queremos atender determinadas regiões do espaço com o drone, porém devemos recarregá-lo em determinados pontos de recarga. Um exemplo é o problema The Fuel Constrained UAV Routing Problem (FCURP)~\cite{sundar2013algorithms}.
			
			\item [Monitoramento, Rastreamento, Filmagem.] Dentre muitos exemplos, esta aplicação pode ter motivações militares de monitaremento de fronteiras ou científica, para filmagem em áreas de floresta. Um problema que trata desse assunto é o Close Enough Travelling Salesman Problem (CETSP)~\cite{coutinho2016branch}. No CETSP, o caixeiro (no caso o drone) deve visitar uma região específica que contenha o vértice e não o próprio vértice. Em várias aplicações dessa categoria os UAVs não precisam chegar ao local de destino. Assim, o caminho dele, bem como a energia de sua bateria, podem ser otimizados. Quando o drone está equipado com um sensor, ele pode operar com sucesso a uma certa distância do alvo.
		\end{description}
		\item [VRP com drones (VRP-D).] Classificações para VRP-D são apresentadas abaixo:
		\begin{description}
			\item [Entrega-Transporte (\textit{Delivery-Transportation}).] Semelhante ao descrito anteriormente, porém agora os drones e os caminhões podem fazer parte de um VRP. Exemplo, Delivery por n-Trucks e m-Drones, uma frota de caminhões equipados com drones entregam bens aos clientes~\cite{wang2017vehicle}.
			
			\item [Entrega por Drones.] Entrega feita apenas por drones, o que é desafiador, devido a múltiplas características operacionais como operações ``multi-trip'', planejamento de recarga e cálculo de consumo de energia. Exemplo: Multi-Trip Drone Routing Problem (MTDRP)~\cite{cheng2018formulations}, considera a influência da carga útil e da distância na duração do voo.
			
			\item [Monitoramento, Vigilância.] Podemos usar uma frota de drones para combater incêndios florestais~\cite{chen2018adaptive}, que são uma classe de desastres perigosos que são muito difíceis de combater. Especialmente incêndios em terrenos montanhosos ao usar equipamentos tradicionais. Dada a rapidez dos incêndios florestais, a atribuição adaptativa e dinâmica de tarefas de combate a incêndios para drones é de grande importância.  
		\end{description}
	\end{description}
\end{description}



\section{Definição do Problema Escolhido}\label{sec:problem}

Dadas as definições encontradas na literatura, escolhemos atacar um problema TSP-D, mais especificamente o Delivery-Transportation 1-Truck 1-Drone, que chamaremos apenas de TSP-D neste relatório. Nele, optamos por considerar que o caminhão nunca está parado, então em um certo nó, lançamos o drone, que estava no caminhão, para realizar uma entrega e retornar em um ponto de encontro, onde o caminhão passa e coleta o drone. Dessa forma, temos dois conjuntos, um conjunto de pontos atendidos pelo caminhão $TD$ e um conjunto de pontos atendidos pelo drone $DD$, a solução é formada pelos dois conjuntos $sol = \left\{TD, DD\right\}$, sendo que as entregas feitas pelos drones têm que respeitar a capacidade e bateria do drone e são indicadas na forma de tripla ${<}i, j, k{>}$, sendo $i$ o ponto de lançamento, $j$ o ponto atendido pelo drone e $k$ o ponto de encontro. Imagens para ilustrar um tour ótimo para o TSP e para o TSP-D são apresentadas nas Figuras~\ref{fig:tsp} e~\ref{fig:tspd} retiradas do artigo de Quang Minh et al.~\cite{ha2018min}.\par
Mais formalmente, dado um grafo $G = \left(V, A\right)$, com $V = \left\{0, 1,...,n,n+1\right\}$, sendo $0$ e $n+1$ o mesmo depósito, $N = \left\{1,...,n\right\}$, queremos encontrar uma rota que comece com o caminhão e o drone em $0$ e termine com os dois em $n+1$ de forma a minimizar 
$$
C1 \sum_{i \in V_L} \sum_{j \in V_R: i \neq j} d_{ij} x_{ij} + C2 \sum_{i \in V_L} \sum_{j \in N, i \neq j} \sum_{k \in V_R: <i, j, k> \in P} \left(d'_{ij} + d'_{jk}\right) y_{ijk} \enspace,
$$
sendo $C1$ os custos operacionais do caminhão, $C2$ os custos operacionais do drone, $V_L = \left\{0,1,...,n\right\}$, $V_R = \left\{1,2,...,n+1\right\}$, $x_{ij} \in \left\{0, 1\right\}$, $y_{ijk} \in \left\{0, 1\right\}$, $d_{ij}$ a distância percorrida pelo caminhão entre os pares de vértices $i, j$ e $d'_{ij}$ a distância percorrida pelo drone entre $i, j$. Uma solução deve respeitar as seguintes restrições: 
\begin{itemize}
	\item Cada cliente é atendido apenas uma vez pelo caminhão ou pelo drone.
	\item As entregas feitas pelo drone têm que respeitar sua capacidade e bateria.
	\item As entregas feitas pelo drone devem ser compatíveis com a rota do caminhão, isto é, para cada ponto de lançamento deve existir um ponto de encontro e o drone só pode ser lançado novamente após sua coleta nesse ponto de encontro.
\end{itemize}
  

\begin{figure}[htb!]
 \centering
 \begin{minipage}[c]{.49\linewidth}
  \centering
  \includegraphics[scale=0.7]{tsp-otimo.png}
  \caption{Solução ótima para um TSP. Imagem retirada do artigo~\cite{ha2018min}.}
  \label{fig:tsp}
 \end{minipage}\hfill
 \begin{minipage}[c]{.49\linewidth}
  \centering
  \includegraphics[scale=0.6]{tsp-d-otimo.png}
  \caption{Solução ótima para um TSP-D. Imagem retirada do artigo~\cite{ha2018min}.}
  \label{fig:tspd}
 \end{minipage}
\end{figure}


\section{Algoritmos para o TSP-D}\label{sec:algorithms}
% Escrever sobre os algoritmos estudados até então e citar que vamos adaptar os estudados anteriormente para o TSP-D

Na iniciação científica anterior estudamos e implementamos algoritmos heurísticos e metaheurísticos para o TSP e o VRP, alguns dos quais desejamos adaptar para o TSP-D. Assim, estudamos algumas heurísticas específicas para o TSP-D e buscamos formas de adaptar as ideias estudadas anteriormente para este novo problema. Ao longo desta seção descreveremos melhor os procedimentos que estudamos no trabalho de Quang Minh et al.~\cite{ha2018min}.\par
O principal algoritmo estudado é um \textit{Greedy Randomized Adaptive Search Procedure} (GRASP). %cuja fase de busca é uma \textit{Variable Neighborhood Descent}
A cada iteração do algoritmo, realizamos um procedimento construtivo aleatorizado seguido de uma intensificação, a busca local. O procedimento construtivo é um algoritmo guloso aleatorizado para o TSP seguido de um algoritmo para transformar essa solução do TSP em uma do TSP-D.

\begin{algorithm}[htb!]
  \caption{GRASP}\label{alg:GRASP}
  \begin{algorithmic}[1]
    \Require{Grafo $G = \left(V, A\right)$, funções de distância $d$ e $d'$ e fatores de custo $C1$ e $C2$}
    \Ensure{Uma solução $sol^*$}
    \Function{GRASP}{G}
      \State $sol^* \gets +\infty$
      \While{Critério de parada não atingido} 
       \State $tspTour \gets greedyRandTSP(G)$
       \State $P, C, D \gets makeAuxGraph(tspTour)$
       \State $S \gets solveTspD(P, C, D)$
       \State $sol \gets localSearch(S)$ 
 	  \If {$sol$ for melhor que $sol^*$} 
	      \State $sol^* \gets S$
	  \EndIf
      \EndWhile
      \State \textbf{return} $sol^*$
      \EndFunction
  \end{algorithmic}
\end{algorithm}

No Algoritmo~\ref{alg:GRASP} temos o pseudocódigo do algoritmo estudado. Para satisfazer a função greedyRandTSP podemos implementar diversos algoritmos aleatorizados. Alguns exemplos citados no artigo~\cite{ha2018min} são: % Algoritmos aleatorizados encontrados no artigo.
\begin{description}
	\item [\textit{Nearest Neighbor} Aleatorizado.] Nele escolhemos o vértice inicial aleatoriamente e em cada iteração, escolhemos o próximo vértice a ser inserido de forma pseudoaleatória da Lista Restrita de Candidatos (LRC). Na LRC colocamos os $k$ vizinhos mais próximos do último vértice inserido, com $k$ sendo um parâmetro que indica do algoritmo.
	
	\item [Permutação Aleatória.] O algoritmo escolhe aleatoriamente a ordem de visita dos vértices. 
	
	\item [\textit{Cheapest Insertion} Aleatorizado.] A abordagem consiste em começar com um subtour, ou seja, um tour com um subconjunto de vértices do grafo, e então estender este tour 
inserindo repetidamente os nós restantes até que nenhum outro possa ser adicionado. 
O vértice não visitado $v$ a ser inserido e sua localização entre dois nós consecutivos $(i,j)$ do passeio são selecionados de modo que esta combinação forneça o menor Custo de Inserção (CI).
Este custo é calculado por: $$ CI = d_{iv} + d_{vj} - d_{ij} \enspace.$$
Para adicionar aleatoriedade nessa heurística, a cada iteração, escolhemos um vértice não visitado e sua localização de inserção usando uma LRC.

	%\item [Inserção de Bicos]. Neste algoritmo queremos inserir o máximo de triângulos acutângulos no tour do TSP. Uma forma de realizar esta operação é, a partir de uma escolha inicial aleatória, escolha como próximo vértice a ser visitado o segundo mais próximo do último inserido. 
	
	
	%\item [Retirada de Bicos]. Neste algoritmo queremos remover triângulos acutângulos do circuito do TSP da seguinte forma: selecionamos o vértice mais distante do centróide do grafo e consideramos que este vértice é atendido por um drone, então, recalculamos o centróide. A ideia é formar um TSP nos vértices que sobraram e utilizar os vértices marcados como entrega de drone já na etapa de resolução do TSP-D. As retiradas podem parar de acontecer se nenhum vértice estiver acima de uma margem de distanciamento estipulada.
\end{description}

% Explicar makeAuxGraph, a função que forma um grafo auxiliar e retorna P, C, D, três vetores que contêm informações que já nos dão a solução praticamente.
A função makeAuxGraph constrói um grafo auxiliar orientado a partir do circuito do TSP e retorna os vetores $P$, $C$ e $D$. O grafo auxiliar é constituído pelo circuito do TSP e algumas arestas adicionais. %As arestas adicionais são inseridas cada vez que uma tripla $<i, j, k>$ de entrega por drone válida é detectada. 
O procedimento considera cada par de vértices $(i, k)$, com $i$ estando antes de $k$ no circuito. Fixados $(i, k)$ o algoritmo escolhe um $j$ entre $i$ e $k$ para ser atendido pelo drone de modo a minimizar o custo total de entrega. Então, a tripla ${<}i, j, k{>}$ é inserida no vetor $D$, que indica as possíveis entregas por drone, e uma aresta adicional $a_{ik}$ é inserida no grafo auxiliar. Essa aresta simboliza que o drone é lançado em $i$ e o caminhão realiza entregas em todos os vértices até $k$, com exceção do vértice intermediário $j$. A Figura~\ref{fig:aux_graph} ilustra o grafo auxiliar formado por esse procedimento a partir do circuito da Figura~\ref{fig:tsp}.

\begin{figure}[htb!]
\centering
\includegraphics[scale=0.7]{auxiliar_graph.png}
\caption{As arestas na horizontal correspondem ao circuito original, as outras são as adicionadas por makeAuxGraph. Uma $a_{ik}$ com custo $\infty$ indica que não é possível uma entrega com o drone saindo de $i$ e voltando em $k$. Figura retirada do artigo~\cite{ha2018min}.}
\label{fig:aux_graph}
\end{figure}


Por fim, a função makeAuxGraph encontra o caminho mais curto de cada vértice do grafo auxiliar e armazena no vetor $P$, com $P[n + 1]$ representando o ``caminho'' mais curto que começa e termina no depósito. Os custos de cada caminho são armazenados no vetor $C$.


% SolveTSP pega os vetores P, C, D e "formata" em uma solução do tipo Sol = {TD, DD}.

A partir dos vetores $P$, $C$ e $D$, a função solveTspD transforma o caminho mais curto do grafo auxiliar na solução $sol = \left\{TD, DD\right\}$. Isso ocorre por meio da análise do caminho mais curto $P[n+1]$, que é separado em \textit{Truck Deliveries} (TD) e \textit{Drone Deliveries} (DD) de acordo com as arestas utilizadas. Em particular essa etapa usa vetor de triplas $D$ para identificar as DDs e separar das TDs no caminho mais curto.


% Fase de intensificação

Na fase de intensificação é utilizada uma busca local. As estruturas de vizinhaça apresentadas pelo artigo são \textit{Exchange}, também conhecida como \textit{Swap}, e \textit{Relocation}, adaptadas ao TSP-D. 
No TSP, a estrutura de vizinhança \textit{Exchange} funciona através da seleção de dois vértices da solução inicial e troca de suas posições no circuito. Por exemplo, se realizarmos um \textit{exchange} entre 0 e 3 no tour inicial $\left\{0, 1, 2, 3, 4, 5\right\}$, obtemos o tour $\left\{3, 1, 2, 0, 4, 5\right\}$. 
E a estrutura \textit{Relocation} no TSP, funciona realocando um vértice $x$ para ser predecessor de um vértice $y$. Como exemplo, considerando o tour anterior e $x = 1$ e $y = 4$, obtemos $\left\{0, 2, 3, 1, 4, 5\right\}$.



O \textit{Exchange} para TSP-D realiza as seguintes operações de troca: (1) Um nó em DD com um em TD; (2) Dois nós em DD; (3) Dois nós em TD. Em relação a operação 
(1) atualizamos a tripla ${<}i, j, k{>}$ de DD substituindo o vértice $j$ por $x$ e removemos $x$ de TD, inserindo $j$ na sua posição. A operação (2) é simples, apenas realizamos uma troca entre um $j$ e outro $j'$ nas triplas do vetor DD. Já a operação (3) troca dois vértices do vetor TD, tomando os devidos cuidados para atualizar as triplas de DD se algum dos vértices trocados fizer parte de alguma tripla.



O \textit{Relocation} é dividido em duas operações, \textit{Truck-Relocation} e \textit{Drone-Relocation}. A primeira operação funciona de forma semelhante ao \textit{Relocation} do TSP, pois consideramos que as realocações só podem ocorrer entre vértices exclusivos ao TD, isto é, um vértice atendido pelo caminhão que não participe de nenhuma tripla de DD. Já no \textit{Drone-Relocation} a ideia é transformar um vértice de TD em um de DD, formando uma tripla nova, ou alterar os pontos de lançamento $i$ e coleta $k$ de um vértice $j$. As duas operações podem ser realizadas ao mesmo tempo também. Essa estrutura pode resultar em uma vizinhança que apresente mais \textit{drone deliveries} que a solução inicial.



Outra estrutura própria para TSP-D apresentada pelo artigo é o \textit{Drone Removal}, no qual removemos uma tripla de DD e o vértice $j$ passa a ser atendido pelo caminhão.


%No TSP, a estrutura de vizinhança 2-OPT inverte de posição dois vértices $a$ e $b$ selecionados em um circuito, também invertendo toda subsequência entre $a$ e $b$, levando a troca de apenas duas arestas. Já a estrutura de Realocação apenas troca de posição 

%Podemos pensar em um 2-OPT para TSP-D

%Por fim, na fase de intensificação utilizamos uma VND.  Ela usa várias vizinhanças, em ordem crescente de complexidade. Quando uma estrutura de vizinhança menos complexa alcança o mínimo local, avançamos para uma mais complexa. Se esta conseguir algum resultado que melhore a solução, retornamos à estrutura anterior e continuamos a busca com ela. O algoritmo termina quando alcançamos o mínimo local em todas as estruturas.
%Ainda estamos estudando as estruturas de vizinhança que serão implementadas.


\subsection{Algoritmos Propostos}

A partir desse estudo, propusemos alguns algoritmos que podem trazer melhores resultados para o TSP-D. Em particular, pensamos em algumas heurísticas para a fase construtiva, apresentadas a seguir:

\begin{description}
	\item [Inserção de Bicos.] Neste algoritmo queremos inserir o máximo de triângulos acutângulos no circuito do TSP. Uma forma de realizar esta operação é, a partir de uma escolha inicial aleatória, escolher como próximo vértice a ser visitado o segundo mais próximo do último inserido. A ideia é que esses ``bicos'' levam a boas oportunidades de entrega por drone.
	

	\item [Retirada de Bicos.] Neste algoritmo queremos remover triângulos acutângulos do circuito do TSP da seguinte forma: selecionamos o vértice mais distante do centróide do grafo e consideramos que este vértice é atendido por um drone, então, recalculamos o centróide. A ideia é formar um TSP nos vértices que sobraram e utilizar os vértices marcados como entrega de drone já na etapa de resolução do TSP-D. As retiradas podem parar de acontecer se nenhum vértice estiver acima de uma margem de distanciamento estipulada.
	
	
	\item [Expansão do Grafo Auxiliar.] 
	Queremos expandir a construção do grafo auxiliar com o objetivo de não limitar o vértice $j$ a estar entre $i$ e $k$ no circuito do TSP. Para isso, percorremos os vértices do circuito e, a cada iteração, consideramos que o nó corrente é atendido pelo drone. Escolhemos um ponto $i$ de lançamento e um $k$ de coleta de forma gulosa, respeitando apenas a restrição da posição de $k$ ser maior que a de $i$. Dessa forma, criamos caminhos adicionais que armazenam essa tripla. O restante do grafo auxiliar é construído como antes. No final, obtemos $n$ caminhos adicionais, e então construímos $n$ vetores P, C, D para serem considerados na formulação da solução.
	
	
	%\item [Entregas por Drone para Múltiplos Vértices.] \textcolor{blue}{
%No futuro, gostaríamos de tratar da variante do problema em que o drone pode visitar mais de um vértice a cada lançamento. Para isso, uma ideia simples é permitir que o drone faça uma sequência de entregas entre os vértices de lançamento $i$ e de coleta $k$ na etapa de construção do grafo auxiliar. Porem, o número de conjuntos que podem ser formados no intervalo de $i$ até $k$ cresce exponencialmente. Uma possibilidade é construir um caminho com o drone, utilizando o pensamento guloso de inserir o mais próximo (\textit{Nearest Neighbor}) nos vértices de $i$ até $k$. E depois realizar operações de remoção de um vértice desse tour possível entrega por drone, verificando a variação de custo da nova solução produzida a cada iteração dessas remoções, parando quando a remoção trouxer prejuízo. 
%Dessa forma, teriamos triplas do tipo ${<}i,{j_1, ..., j_w}, k{>}$, com cada um dos $j_w$ estando entre o ponto de lançamento e o de coleta. Daí, o resto das etapas de obtenção da solução do TSP-D seguem de forma semelhante, somente com a semântica da aresta $a_{ik}$ alterada.
%}
	
\end{description}

Na fase de intensificação pensamos em utilizar a metaheurística VND (\textit{Variable Neighborhood Descent}) para substituir a heurística de busca local. A VND usa várias vizinhanças, em ordem crescente de complexidade. Quando uma estrutura de vizinhança menos complexa alcança o mínimo local, avançamos para uma mais complexa. Se esta conseguir algum resultado que melhore a solução, retornamos à estrutura anterior e continuamos a busca com ela. O algoritmo termina quando alcançamos o mínimo local em todas as estruturas. No caso, as estruturas que pensamos em utilizar são as mesmas apresentadas pelo artigo estudado além de uma adaptação da vizinhança 2-OPT para o TSP-D, que queremos propor.

\section{Cronograma}\label{sec:crono}

A Tabela~\ref{tab:cron} contém o cronograma proposto.

\begin{table}[htb!]
\begin{center}
    \caption{Cronograma para as próximas etapas.}
    %\begin{tabular}{ |l*{6}{|c}| }
    \begin{tabular}{ |l*{12}{|c}| }
    \hline
    \multirow{2}{*}{Atividades} & \multicolumn{12}{ |c| }{Meses} \\
    \cline{2-13}
    & ~1 & ~2 & ~3 & ~4 & ~5 & ~6 & ~7 & ~8 & ~9 & 10 & 11 & 12 \\ \hline
    Classificação dos Problemas & $\bullet$ & $\bullet$ & $\bullet$ &&&&&&&&& \\ \hline
    Heurísticas &&&& $\bullet$ & $\bullet$ & $\bullet$ & $\bullet$ &&&&& \\ \hline
    Meta-Heurísticas &&&&&&&& $\bullet$ & $\bullet$ &$\bullet$ & $\bullet$ & $\bullet$\\ \hline
    Implementação & $\bullet$ & $\bullet$ & $\bullet$ & $\bullet$ & $\bullet$ & $\bullet$ & $\bullet$ & $\bullet$ & $\bullet$ & $\bullet$ & $\bullet$ & $\bullet$\\ \hline
    \end{tabular}\label{tab:cron}
\end{center}
\end{table}

O primeiro semestre foi destinado ao estudo e compreensão dos problemas de roteamento envolvendo drones. Destacamos que esse processo levou mais tempo que o esperado, por conta da variedade de problemas e da literatura ser recente, o que acarreta grande quantidade de problemas semelhantes e inconsistências na nomenclatura. Apesar disso, o processo de revisão bibliográfica trouxe um aprendizado importante para o estudante. A partir da compreensão dos problemas, buscamos escolher um problema viável para ser atacado durante o tempo da IC. Decidimos pelo TSP-D, por ele ser próximo ao TSP, ao mesmo tempo que apresenta relevância prática e um desafio interessante. Por conta dos conhecimentos sobre heurísticas e metaheurísticas adquiridas na iniciação anterior, escolhemos aprofundar no artigo de Quang Minh et al.~\cite{ha2018min} que aborda TSP-D com GRASP.\par 
Feita esta escolha, pensamos em estender o trabalho estudado usando um GRASP-VND com novas heurísticas gulosas aleatorizadas. Nas próximas etapas do projeto pretendemos implementar e testar algumas dessas nossas ideias. %resolvemos estudar artigos que tratam de heurísticas específicas para este problema~\cite{ha2018min, sugihara2011path}. Nas próximas etapas do projeto pretendemos implementar e testar essas heurísticas adaptadas. 


\section{Conclusão}\label{sec:concl}

Nos primeiros meses desta iniciação científica buscamos entender a área de roteamento de veículos com drones e enfrentamos algumas dificuldades para entender a nomenclatura desta área. Por isso, partimos para o estudo de surveys que nos ajudaram a entender a variedade e classificação dos problemas, principalmente, o survey de Khoufi et al.~\cite{khoufi2019survey}. Dos problemas encontrados nessa pesquisa, escolhemos o problema TSP-D para atacar.
Em seguida estudamos heurísticas e metaheurísticas específicas deste problema e projetamos novas, as quais pretendemos implementar o mais breve possível. \par
As reuniões com o orientador foram feitas quinzenalmente, com exceção do mês de janeiro, em que foram semanais. Ainda não obtivemos resultados empíricos, porém considero satisfatório a ampla revisão da literatura realizada para entender as classificações de problemas de roteamento com drones, pois adquiri bastante conhecimento em problemas NP-Difíceis e pesquisa científica de artigos e surveys.\par 



\bibliographystyle{plain}
\bibliography{referencias}


\end{document}
